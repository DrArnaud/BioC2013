%
% NOTE -- ONLY EDIT THE .Rnw FILE!!!  The .tex file is
% likely to be overwritten.
%
% Sweave("~/git/BioC2013/BioC2013_SS.Rnw")
% Stangle(system.file("BioC2013_SS.Rnw",package="ChipSeq"))
%\VignetteIndexEntry{}
% \VignetteDepends{PICS,rGADEM,MotIV,PING}
%\VignetteKeywords{}
%\VignettePackage{}

\documentclass[12pt]{article}

\usepackage{amsmath,pstricks}
% \usepackage[authoryear,round]{natbib}
\usepackage{hyperref}
\usepackage{amsmath}
\usepackage[super,numbers,round,sort&compress]{natbib}

\textwidth=6.2in
\textheight=8.5in
%\parskip=.3cm
\oddsidemargin=.1in
\evensidemargin=.1in
\headheight=-.3in

\newcommand{\scscst}{\scriptscriptstyle}
\newcommand{\scst}{\scriptstyle}


\newcommand{\Rfunction}[1]{{\textit{#1}}}
\newcommand{\Robject}[1]{{``#1''}}
\newcommand{\Rpackage}[1]{\texttt{#1}}
\newcommand{\Rmethod}[1]{\textit{#1}}
\newcommand{\Rfunarg}[1]{{`#1'}}
\newcommand{\Rclass}[1]{{\textit{#1}}}

\textwidth=6.2in

 
\usepackage{Sweave}
\begin{document}
%\setkeys{Gin}{width=0.55\textwidth}

\newtheorem{Exercise}{Exercise}[section]

\title{chip-seq data analysis \\ 2013 BioConductor LabSession: ``A Bioconductor pipeline for the analyis of ChIP-Seq experiments.''}
\author{Xuekui Zhang, Sangsoon Woo and Arnaud Droit}

\maketitle

\section{Introduction}

This package \Rclass{ChipSeq} provides all necessary data for the chip-seq sections of the 2013 BioConducor Labsession. This vignette also includes all commands that we will be used throughout the lab sessions \cite{Mercier:2011p5429,Zhang:2010p3139}.

\part{ChIP-Seq for Transcription Binding sites}
\section{Data Input}
For your convenience, the experimental data required in this package have already been pre-formatted and can simply be loaded with the following commands:
\begin{Schunk}
\begin{Sinput}
> library(ChipSeq)